\section{Application}

\subsection{AEAD}

	Authenticated Encryption (AE) is an algorithm form that operates the encryption and the authentication at the same time. Typical AE contains

	\begin{itemize}
		\item
			\textbf{Encrypt-then-MAC (EtM)}
			First, plain text $P$ is encrypted using key $K_1$, yielding cipher text $C$. Then $C$ is hashed using another key $K_2$, giving a MAC $M$. Finally, $\{C,M\}$ is sent.
		\item
			\textbf{MAC-then-Encrypt (MtE)}
			First, plain text $P$ is hashed using key $K$, giving a MAC $M$. Then $\{P,M\}$ is encrypted using the same key $K$, yielding cipher text $C$. Finally, $C$ is sent.
	\end{itemize}

	MtE is proved to be vulnerable, and EtM is suggested to use in Transport Layer Security (TLS) and Datagram Transport Layer Security (DTLS) \cite{etm}.

	Authenticated encryption with associated data (AEAD) is a variant of AE that allows a recipient to check the integrity of both the encrypted and unencrypted information in a message \cite{nist-18}. AEAD adds associated data (AD) that is not encrypted to the cipher and to the context where it is supposed to appear. If the cipher appears in a different context, the decryption can fail even with the correct key. So with AEAD, it protects against the operation of copying and pasting a cipher text into another context. AEAD is very similar to EtM, except that AD is not hashed or encrypted and should be known as input, MAC is a hashed value and can be sent together with cipher.

\subsection{TLS 1.2}

	Message Authentication Code (MAC) is adopted in Transport Layer Security (TLS) Protocol. TLS is a widely used protocol that allows client/server applications to communicate over the Internet in a way that is to prevent eavesdropping, tampering, and message forgery \cite{tls-1-2}. A basic TLS 1.2 handshake \cite{tls-1-2} goes as

	\begin{enumerate}[noitemsep]
		\item $\rightarrow$ Client Hello
		\item $\leftarrow$ Server Hello
		\item $\leftarrow$ Server Certificate
		\item $\leftarrow$ Server Key Exchange
		\item $\leftarrow$ Server Hello Done
		\item $\rightarrow$ Client Certificate
		\item $\rightarrow$ Client Key Exchange
		\item $\rightarrow$ Client Change Cipher Spec
		\item $\rightarrow$ Finished
		\item $\leftarrow$ Server Change Cipher Spec
		\item $\leftarrow$ Finished
		\item $\leftrightarrow$ Application Data
	\end{enumerate}

	In TLS 1.2, Message Authentication Code (MAC) is checked in the finished stage. All previous exchanged data are calculated with the master secret to get a key-hashed MAC and checked on both client and server side. Supported MAC algorithms include \emph{HMAC-MD5}, \emph{HMAC-SHA1}, \emph{HMAC-SHA2} and \emph{AEAD} \cite{tls-1-2}.

\subsection{TLS 1.3}
	TLS 1.3 has changed greatly in MAC usage since TLS 1.2. A basic TLS 1.3 handshake \cite{tls-1-3} goes as

	\begin{enumerate}[noitemsep]
		\item $\rightarrow$ Client Hello
		\item $\rightarrow$ Client Key Exchange
		\item $\leftarrow$ Server Hello
		\item $\leftarrow$ Server Key Exchange
		\item $\leftarrow$ Server Parameters
		\item $\leftarrow$ Server Certificate
		\item $\leftarrow$ Server Certificate Verify
		\item $\leftarrow$ Finished
		\item $\rightarrow$ Client Certificate
		\item $\rightarrow$ Client Certificate Verify
		\item $\rightarrow$ Finished
		\item $\leftrightarrow$ Application Data
	\end{enumerate}

	In TLS 1.3, the only supported MAC algorithm for the symmetric encryption is \emph{AEAD} \cite{tls-1-3}, and all \emph{HMAC} hash functions are removed. However, \emph{HMAC} is still used in finished stage to check the certificate and certificate verify information. The key is instead generated from the master secret with a key derivation function \cite{tls-1-3}.