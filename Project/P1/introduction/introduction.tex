\section{Introduction}

\subsection{Background}

	Digital data security has long been a popular issue with the revolution of network connections and data communications. During the data transfer process, digital information can easily be added, deleted and modified by everyone. Therefore, when a data receiver gets a piece of information from the network, the information is not guaranteed to be the same message as the sender sends. To verify one message, one may want to check its integrity, authenticity and non-repudiation.

	\begin{itemize}
		\item
			\textbf{Integrity:}
			Completeness, consistency and accuracy of data \cite{integrity}. Integrity can be ensured if content or time of data has not been accidentally modified.
		\item
			\textbf{Authenticity:}
			Originality or genuineness of data. Whether the message originates from the sender.
		\item
			\textbf{Non-repudiation:}
			If the message is sent to a third party, whether the third party can also verify integrity and authenticity of data \cite{non-repu}.
	\end{itemize}

	Message authentication code (MAC) is one of the major methods to check both integrity and authenticity, but not the non-repudiation of message. MAC is appended to the message by the sender, and receiver is able to use MAC to tell whether the message has accidentally changed or is sent by a third party who does not own the key. MAC is now widely used in Internet security and unreliable data storage.

\subsection{Methods}

	MAC achieves the authentication by creating a short piece of information given the original message and one symmetric key. The process flow diagram is shown in Figure \ref{fig:mac}.

	\begin{enumerate}
		\item
			\textbf{Requirements}

			MAC works based on a symmetric key. The key must be shared before MAC between sender and receiver in person.
		\item
			\textbf{Sender}

			The sender first prepared the original message. Then with the symmetric key, the sender combines the message and the key using a MAC algorithm.

			The MAC algorithm is implemented in two ways: 1) Hash-based Message Authentication Code (HMAC), which uses a hash function such as \emph{MD5}, \emph{SHA-1} and \emph{SHA-2}, to generate a key-hashed checksum. 2) Block Cipher-based Message Authentication Code (CMAC), which uses block cipher algorithms such as \emph{AES} and \emph{DES} to get cipher text of fixed length.

			Finally, the sender appends MAC to the message and sends them in a bundle to the receiver.
		\item
			\textbf{Receiver}

			The receiver gets the message and MAC from the network. First, the same symmetric key shared before is applied on the MAC algorithm and the message to generate a MAC from the receiver side. The MAC algorithm must be the same one in the sending process, which can be agreed on when sharing the key.

			Next, the receiver simply compares the two MAC. If two MAC are the same, then the receiver can confirm the integrity and authenticity of the message; Otherwise he cannot, which means that data content has been either accidentally modified or tampered by someone else.
	\end{enumerate}

	Since MAC is based on the symmetric key, the integrity can be ensured if the key is not leaked to someone other than the sender and the receiver, and the security level should be same as that of the MAC algorithm. Similarly, authenticity can also be verified from the receiver side. However, MAC does not guarantee non-repudiation. If a third party receives the message, he cannot tell whether the message originates from the sender or the receiver. Therefore, if the receiver manipulates a message and claims it to be sent from the sender, the sender cannot deny, since they have the same key. To provide non-repudiation, one need to use Digital Signature \cite{dig-sig} instead of MAC.
	\begin{figure}[H]
		\centering
		\includesvg[width=\textwidth]{introduction/mac}
		\caption{Message Authentication Code}\label{fig:mac}
	\end{figure}