\documentclass[12pt]{article}

% Language setting
% Replace `english' with e.g. `spanish' to change the document language
\usepackage[english]{babel}

% Set page size and margins
% Replace `letterpaper' with`a4paper' for UK/EU standard size
% \usepackage[letterpaper,top=2cm,bottom=2cm,left=3cm,right=3cm,marginparwidth=1.75cm]{geometry}

% Useful packages
\usepackage[margin=2.54cm]{geometry}
\usepackage{amsmath}
\usepackage{amsfonts}
\usepackage{amssymb}
\usepackage{graphicx}
\usepackage{hyperref}
\usepackage{multirow}
\usepackage{csquotes}
\usepackage[section]{placeins}
\usepackage{caption}
\usepackage{subcaption}
\usepackage{indentfirst}
\usepackage{enumerate}
\usepackage{float}
\usepackage{algorithm}
\usepackage{algpseudocode}
\title{VE475 Homework 5}
\author{Yiwen Yang}

	
\begin{document}
\date{}
\maketitle

\section*{Ex.1 - RSA Setup}

	\begin{enumerate}
		\item
			Since $n=pq$, the probability of $gcd(m,n)\not=1$ is $\frac{1}{p}+\frac{1}{q}-\frac{1}{n}$ which is very small, so $n$ is very likely to be coprime with $m$.
		\item
			\begin{enumerate}
				\item
					According to Euler Theorem
					$$m^{\varphi(p)}\equiv 1 \bmod p$$
					Then
					$$m^{k}\equiv m^{a\cdot \varphi(n)}\equiv m^{a\cdot \varphi(p)\varphi(q)}\equiv  1 \bmod p$$
					And similarly
					$$m^{k}\equiv 1 \bmod q$$
				\item
					If $gcd(m,p)=1$, then obviously
					$$m^{k+1}\equiv m^k\cdot m\equiv m \bmod p$$
					If $gcd(m,p)\not=1$, \textit{i.e.}, $gcd(m,p)=p$.
					$$m\equiv 0 \bmod p$$
					$$m^{k+1}\equiv 0\equiv m \bmod p$$
					So for any $m$, $m^{k+1}\equiv m \bmod p$, and similarly $m^{k+1}\equiv m \bmod q$.
			\end{enumerate}
		\item
			\begin{enumerate}
				\item
					Since we have
					$$ed\equiv 1 \bmod \varphi(n)$$
					Then
					$$ed=a\varphi(n)+1$$
					$$m^{ed}\equiv m^{a\varphi(n)+1}\equiv m \bmod n$$
					For arbitrary $m$.
				\item
					The decryption of RSA calculates
					$$c^d\equiv m^{ed}\equiv m \bmod n$$
					Since this process holds true for arbitrary $m$ as proved above, there is no such need that $gcd(m,n)=1$.
			\end{enumerate}
	\end{enumerate}

\section*{Ex.2 - RSA Decryption}

	Factorizing 11413 gets $101\times113$, then $\varphi(11413)=100\times112=11200$. Use Extended Euclidean Algorithm to find $d=e^{-1}\bmod \varphi(n)$
	\begin{align*}
		11200&=7467\times1+3733\\
		7467&=3733\times2+1\\
		1&=7467-2\times3733\\
		&=7467-2\times(11200-7467)\\
		&=3\times7467-2\times11200
	\end{align*}

	Yields $d=3$. Then
	$$m=c^{d}\bmod n=5859^3 \bmod 11413=1415$$

	So the plain text is 1415.

\section*{Ex.3 - Breaking RSA}

	\begin{enumerate}
		\item
			Short keys are fast to generate and fast to encrypt/decrypt as well.
		\item
			First introduce the continued fraction representation. Denote
			$$\langle q_0,q_1,\ldots,q_m\rangle=q_0+\frac{1}{q_1+\frac{1}{\ldots\frac{1}{q_{m-1}+\frac{1}{q_m}}}}$$
			To extract continued fraction from rational, just take the inverse of the rational and minus the maximum integer repeatedly. To calculate the rational in the order of $q_0$ to $q_m$, define
			$$\frac{n_i}{d_i}=\langle q_0,q_1,\ldots,q_m\rangle,gcd(n_i,d_i)=1,i=0,1,\ldots,m$$
			And
			\begin{align*}
				n_0&=q_0\\
				n_1&=q_0q_1+1\\
				n_i&=q_in_{i-1}+n_{i-2}\\\\
				d_0&=1\\
				d_1&=q_1\\
				d_i&=q_id_{i-1}+d_{i-2}
			\end{align*}
			Then calculating $f=n_m/d_m$ gives the result.

			Given some $f^{'}$ such that $f^{'}=f(1-\delta)$, we can estimate $f$ from $f^{'}$ when $\delta$ is small enough, and the following algorithm is used.
			\begin{algorithm}[H]
				\caption{Reconstruct $f$}
				\textbf{Input:} Rational number $f^{'}$\\
				\textbf{Output:} Array of possible rational number $f$
				\begin{algorithmic}
					\State $i \gets 0$
					\State $q \gets \text{ContinuedFractionFormOf}(f^{'})$
					\State $r \gets Array.empty$
					\For{$i \gets 0$ until $q$.length}
						\State $c \gets \text{RationalFormOf}(q[0],q[1],\ldots,q[i])$
						\If{$i$ is Even}
							\If{$c$ Equals to RationalFormOf$(q[0],q[1],\ldots,q[i-1],q[i]+1)$}
								\State $r+=c$
							\EndIf
						\ElsIf{$i$ is Odd}
							\If{$c$ Equals to RationalFormOf$(q[0],q[1],\ldots,q[i-1],q[i])$}
								\State $r+=c$
							\EndIf
						\EndIf
					\EndFor
					\Return{$r$}
				\end{algorithmic}
			\end{algorithm}
			Next go back to Wiener's Attack, write
			\begin{align*}
				ed&=k\varphi(n)+1\\
				&=k(n-p-q-1)+1\\
				\frac{ed}{n}&=k(1-\delta)\\
				\frac{e}{n}&=\frac{k}{d}(1-\delta)
			\end{align*}
			Where $\delta=(p+q-1-1/k)/n$. Then using the above method, we can derive a list of possible $k$ and $d$ given $e$ and $n$. To verify and get correct keys, solve Eq.\ref{eq:fact} to see whether there are solutions to integer $p$ and $q$.
			\begin{equation}
				\label{eq:fact}
				x^2-(n-\frac{ed-1}{k}+1)x+n=0
			\end{equation}
		\item
			The length of the key should be larger than $\frac{1}{3}n^{\frac{1}{4}}$.
		\item
			The result is found as $n=12457\times25523,d=41$. Source code can be seen in folder \textbf{ex3}.
	\end{enumerate}

\section*{Ex.4 - Programming}

	See in folder \textbf{ex4}.

\section*{Ex.5 - Simple Questions}

	\begin{enumerate}
		\item
			Choose a proper $r$ such that calculating $r^e\cdot c\bmod n$ will give $r\cdot m$, which means that we can indirectly decrypt the cipher derived from another one.
		\item
			No, since doubling the encryption does not add to the difficulty of factorizing numbers, so if it takes the attacker some certain time to hack the first key, it will just take another same amount of time to hack the other one, which does not make a big difference.
		\item
			$$187722^2-516107^2\times 4\equiv0\bmod n$$
			$$(187722+516107\times2)(187722-516107\times2)\equiv0\bmod n$$
			$$1219936\times844492\equiv0\bmod n$$
			$$64866\times844492\equiv0\bmod n$$
			$$2^3\times3\times19\times569\times211123\equiv0\bmod n$$
			Obviously 2, 3, 19 are not factors of 642401, and 569 is. So $642401=569\times1129$.
		\item
			The process is exactly the same except that $\varphi(n)=(p-1)(q-1)(r-1)$. If $n$ has the same length, then with three factors the attack will be easier since each factor is smaller, so the security level is lower than two factors.
		\item
			For all $q\mid 96$, $q=2,3$, test $\alpha$ from 2 such that $\alpha^{96/q}\not\equiv 1\bmod 97$, and the smallest $\alpha$ is 5. So the smallest generator of $U(\mathbb{Z}/97\mathbb{Z})$ is 5.
		\item
			\begin{enumerate}
				\item
					For all $q\mid 100$, $q=2,5$, test $\alpha=2$ and get $\alpha^{50}\equiv100\not\equiv 1\bmod 100$, $\alpha^{20}\equiv95\not\equiv 1\bmod 100$, so 2 is a generator of $U(\mathbb{Z}/101\mathbb{Z})$.
				\item
					$\log_2 24=\log_2 3+\log_2 8=69+3=72$
				\item
					$\log_2 24=\log_2 125=3\log_2 5=72$
			\end{enumerate}
		\item
			For all $q\mid 136$, $q=2,17$, test $\alpha=3$ and get $\alpha^{68}\equiv136\not\equiv 1\bmod 137$, $\alpha^{8}\equiv122\not\equiv 1\bmod 137$, so 3 is a generator of $U(\mathbb{Z}/137\mathbb{Z})$.

			Then $x=\log_3 11=\log_3 44-2\log_3 2=-14=122$.
		\item
			\begin{enumerate}
				\item
					$6^5\equiv 3^2\times6\equiv10\bmod 11$
				\item
					For all $q\mid 10$, $q=2,5$, test $\alpha=2$ and get $\alpha^{5}\equiv10\not\equiv 1\bmod 11$, $\alpha^{2}\equiv4\not\equiv 1\bmod 11$. So 2 is a generator of $U(\mathbb{Z}/11\mathbb{Z})$.
				\item
					$(2^5)^x\equiv6^5\equiv-1\bmod11$, so $x$ should be odd.
			\end{enumerate}
	\end{enumerate}

\section*{Ex.6 - DLP}

	\begin{enumerate}
		\item
			\begin{align*}
				3^{16x}&\equiv2^{16}\equiv-1\bmod65537\\
				3^{32x}&\equiv1\bmod65537\\
				3^{65536}&\equiv1\bmod65537
			\end{align*}
			So $65536\mid 32x$, $65536\nmid 16x$ $\Rightarrow$ $2048\mid x$, $4096\nmid x$.
		\item
			$x$ satisfies $x=2048(2k+1)$, so there are 16 possible choices. Applying modular exponentiation, find that $3^{2048}\equiv65529\equiv-8\bmod65537$. Try $k$ from 0 to 15.
			$$(-8)^7\equiv32\bmod65537$$
			$$(-8)^11\equiv-2\bmod65537$$
			$$(-8)^17\equiv-2\bmod65537$$
			$$(-8)^23\equiv-32\bmod65537$$
			$$(-8)^27\equiv2\bmod65537$$
			So $x=2048\times27=55296$.
		\item
			The only factor of 65536 is 2, it is easy to apply Pohlig-Hellman algorithm.

			Since $x=2048(2k+1)$, then write $x=2^{11}+c_{12}2^{12}+c_{13}2^{13}+c_{14}2^{14}+c_{15}2^{15}$.
			$$(2\times3^{-1})^{32768/2^{12}}=(3^{32768})^{c_{12}}\Rightarrow c_{12}=1$$
			$$(2\times3^{-1})^{32768/2^{13}}=(3^{32768})^{c_{13}}\Rightarrow c_{12}=0$$
			$$2^{32768/2^{14}}=(3^{32768})^{c_{14}}\Rightarrow c_{14}=1$$
			$$(2\times3^{-1})^{32768/2^{15}}=(3^{32768})^{c_{15}}\Rightarrow c_{15}=1$$
			So $x=2^{11}+2^{12}+2^{14}+2^{15}=55296$.
		\item
			Such primes $p$ has only one factor for $p-1$, and according to above methods, DLP is a lot easier to solve. It is not safe in a cryptographic context.
	\end{enumerate}

\end{document}