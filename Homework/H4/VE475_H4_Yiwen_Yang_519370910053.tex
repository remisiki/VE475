\documentclass[12pt]{article}

% Language setting
% Replace `english' with e.g. `spanish' to change the document language
\usepackage[english]{babel}

% Set page size and margins
% Replace `letterpaper' with`a4paper' for UK/EU standard size
% \usepackage[letterpaper,top=2cm,bottom=2cm,left=3cm,right=3cm,marginparwidth=1.75cm]{geometry}

% Useful packages
\usepackage[margin=2.54cm]{geometry}
\usepackage{amsmath}
\usepackage{amsfonts}
\usepackage{amssymb}
\usepackage{graphicx}
\usepackage{hyperref}
\usepackage{multirow}
\usepackage{csquotes}
\usepackage[section]{placeins}
\usepackage{caption}
\usepackage{subcaption}
\usepackage{indentfirst}
\usepackage{enumerate}
\title{VE475 Homework 4}
\author{Yiwen Yang}


\begin{document}
\date{}
\maketitle

\section*{Ex.1 - Euler's Totient}

	\begin{enumerate}
		\item
			Euler's totient function $\varphi(p^k)$ counts the invertible elements in $\mathbb{Z}/p^k\mathbb{Z}$, that is to find number of elements that are coprime with $p^k$. Since $p$ is prime, then to have a common divisor with $p^k$ is to have a factor of $p$. In this set, $p,2p,\ldots,(p^{k-1}-1)p$ are not coprime with $p^k$, which contains $p^{k-1}-1$ elements. So $\varphi(p^k)=p^k-1-(p^{k-1}-1)=p^{k-1}(p-1)$.
		\item
			Given $m$ is coprime with $n$, according to Chinese Remainder Theorem, there exists a ring isomorphism between $\mathbb{Z}/mn\mathbb{Z}$ and $\mathbb{Z}/m\mathbb{Z}\times\mathbb{Z}/m\mathbb{Z}$. Since there is a bijection between two sets, the invertible element count should be the same, so $\varphi(mn)=\varphi(m)varphi(n)$.
		\item
			Factorize $n$ and let
			\begin{align*}
				n=\prod_{i}p_i^{k_i}
			\end{align*}
			Where $p_i$ are all different primes. Then
			\begin{align*}
				\varphi(n)&=\prod_{i}\varphi(p_i^{k_i})\\
				&=\prod_{i}p_i^{k_i-1}(p_i-1)\\
				&=\prod_{i}p_i^{k_i}(1-\frac{1}{p_i})\\
				&=n\prod_{i}(1-\frac{1}{p_i})
			\end{align*}
		\item
			7 is coprime with 1000, and $\varphi(1000)=1000\times (1-\frac{1}{2})\times (1-\frac{1}{5})=400$. According to Euler's Theorem, $7^{400}\equiv 1 \bmod 1000$. Then $7^{803}\equiv 7^{3} \equiv 343 \bmod 1000$, so the last three digits of $7^{803}$ is 343.
	\end{enumerate}

\newpage{}
\section*{Ex.2 - AES}

	\begin{enumerate}
		\item
			128 bits of 1
		\item
			$K(5)=K(1)\oplus K(4)$
		\item
			To \texttt{xor} a number with all ones is to bit-wise \texttt{not} the number. Since $K(0)=K(1)=K(2)=K(3)=111\ldots1$, then
			\begin{align*}
				K(5)&=K(1)\oplus K(4)=\overline{K(4)}\\
				K(6)&=K(2)\oplus K(5)=\overline{K(5)}\\
				K(7)&=K(3)\oplus K(6)=\overline{K(6)}\\\\
				K(9)&=K(5)\oplus K(8)=\overline{K(6)}\\
				K(10)&=K(6)\oplus K(9)\\
				&=K(6)\oplus K(5)\oplus K(8)\\
				&=\overline{K(5)}\oplus K(5)\oplus K(8)\\
				&=K(1)\oplus K(8)\\
				&=\overline{K(8)}\\\\
				K(11)&=K(7)\oplus K(10)\\
				&=\overline{K(6)}\oplus K(6)\oplus K(9)\\
				&=K(1)\oplus K(9)\\
				&=\overline{K(9)}\\\\
			\end{align*}
	\end{enumerate}

\section*{Ex.3 - Simple Questions}

	\begin{enumerate}
		\item
			\textbf{ECB mode}
				All blocks are decrypted independently, so if one of the cipher blocks is corrupted, only the corresponding decrypted plain text block will be incorrect.
			\\
			\textbf{CBC mode}
				Each cipher block is used both in current block of decryption and also in next block of decryption as IV (xor with next deciphered block), so two output blocks will be broken if one of the cipher blocks is corrupted.
		\item
			If IV is incremented by 1 each time, it is very likely that a list of successive plain text messages begin with the same word (e.g. "SSID = 123456", "SSID = 001122", $\ldots$). Then Eve may observe a same pattern in all first blocks, or two different messages share the same headings, which is leaking much information. If Eve can apply CPA, he may find out which part of bits changes accordingly and predict the first block of plain text when next cipher message is sent by others.
		\item
			29 is prime, and 28 has two prime factors 2 and 7. Test
			\begin{align*}
				2^{28/2}=2^{14}\equiv12^2\equiv28&\neq1\bmod 29\\
				2^{28/7}=2^{4}\equiv16&\neq1\bmod 29
			\end{align*}
			So 2 is a generator of $U(\mathbb{Z}/29\mathbb{Z})$.
		\item
			\begin{align*}
				(\frac{1801}{8191})&=(\frac{987}{1801})\\
				&=(\frac{3}{1801})(\frac{7}{1801})(\frac{47}{1801})\\
				&=(\frac{1}{3})(\frac{2}{7})(\frac{15}{47})\\
				&=(\frac{3}{47})(\frac{5}{47})\\
				&=(\frac{2}{3})(\frac{2}{47})\\
				&=-1
			\end{align*}
		\item
			The number of solutions to the equation depends on $b^2-4ac$.
			\begin{enumerate}
				\item
					$b^2-4ac=0$

					The equation has only one solution, and $1+(\frac{0}{p})=1$.
				\item
					$b^2-4ac>0$

					The equation has two different solutions.
					\begin{align*}
						\frac{-b\pm\sqrt{b^2-4ac}}{2a}&\equiv x \bmod p\\
						b^2-4ac&\equiv (2ax+b)^2 \bmod p
					\end{align*}
					So $b^2-4ac$ is a square mod $p$, $1+(\frac{b^2-4ac}{p})=2$.
				\item
					$b^2-4ac<0$

					The equation has zero solutions, and $b^2-4ac$ is not a square mod $p$, $1+(\frac{b^2-4ac}{p})=0$.
			\end{enumerate}
			Then we can conclude that number of solutions mod $p$ is $1+(\frac{b^2-4ac}{p})$.
		\item
			Given $p,q$ are primes, then
			\begin{align*}
				n^{p-1}&\equiv1\bmod p\\
				n^{q-1}&\equiv1\bmod q\\
				cp&+dq=1
			\end{align*}
			Since $q-1\mid p-1$, so
			\begin{align*}
				n^{p-1}\equiv1\bmod q
			\end{align*}
			According to Chinese Remainder Theorem, if $gcd(n,pq)=1$, then
			\begin{align*}
				n^{p-1}\equiv cp+dq\equiv1\bmod pq
			\end{align*}
		\item
			\begin{enumerate}
				\item
					If $(\frac{-3}{p})=1$, since $p$ is prime, then $(\frac{-1}{p})(\frac{3}{p})=(-1)^{\frac{p-1}{2}}(\frac{3}{p})=1$.

					When $p\equiv1\bmod4$, $(\frac{-1}{p})=1$. Then $(\frac{3}{p})=(\frac{p}{3})=1$, which implies $p\equiv1\bmod3$.

					When $p\equiv3\bmod4$, $(\frac{-1}{p})=-1$. Then $(\frac{3}{p})=-(\frac{p}{3})=-1$, which implies $p\equiv1\bmod3$.

					So $p\equiv1\bmod3$.
				\item
					If $p\equiv1\bmod3$, then $(\frac{p}{3})=1$.

					When $p\equiv1\bmod4$, $(\frac{3}{p})=(\frac{p}{3})=1$. Then $(\frac{-3}{p})=(\frac{-1}{p})(\frac{3}{p})=1$.

					When $p\equiv3\bmod4$, $(\frac{3}{p})=-(\frac{p}{3})=-1$. Then $(\frac{-3}{p})=(\frac{-1}{p})(\frac{3}{p})=1$.

					So $(\frac{-3}{p})=1$.
			\end{enumerate}
			So $(\frac{-3}{p})=1$ if and only if when $p\equiv1\bmod3$.
		\item
			If $(\frac{a}{p})=1$, then $a^{(p-1)/2}\equiv1\bmod p$, and 2 is a factor of $p-1$. Then $a$ cannot be a generator of $p$.
	\end{enumerate}

\section*{Ex.4 - Prime vs. Irreducible}

	\begin{enumerate}
		\item
			Suppose that $p$ is not irreducible, \textit{i.e.}, $p=ab$ where $a,b$ are non-zero, non-unit, non-invertible different elements. Then obviously, $ab\mid ab$, so according to (*), this implies $ab\mid a$ or $ab \mid b$. If $ab \mid a$, then $b=1$, which contradicts. Similarly, $ab \mid b$ also not holds. So (*) indicates that $p$ is irreducible.
		\item
			Suppose that $a$ is neither 1 nor $p$, then $a$ is a factor of $p$, which means that $p$ is irreducible, which contradicts. So irreducible indicates (**).
		\item
			(**) indicates that if $p$ is prime (**), then it is irreducible. Suppose $p\nmid x$ and $p\nmid y$, then $p\nmid xy$, so (*) must hold.
		\item
			From (1) and (2) we can conclude that (*) implies (**), and from (3) (**) implies (*). So (*) and (**) are equivalent.
	\end{enumerate}

\section*{Ex.5 - Primitive Root Mod 65537}

	\begin{enumerate}
		\item
			$(\frac{3}{65537})=(\frac{2}{3})=-1$
		\item
			$(\frac{3}{65537})=3^{32768}\equiv-1\not\equiv1\bmod65537$
		\item
			Since 65537 is prime, and the factor of 65536 is 2, which satisfies that $3^{(65537-1)/2}\not\equiv1\bmod65537$. So 3 is a generator of 65537.
	\end{enumerate}




% \begin{figure}[!hbtp]
% \centering
% \includegraphics[width=0.7\textwidth]{Bootstrap_factor.png}
% \caption{\label{fig:1}RMSE against bootstrap sample size.}
% \end{figure}

\end{document}