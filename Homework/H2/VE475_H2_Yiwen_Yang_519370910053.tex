\documentclass[12pt]{article}

% Language setting
% Replace `english' with e.g. `spanish' to change the document language
\usepackage[english]{babel}

% Set page size and margins
% Replace `letterpaper' with`a4paper' for UK/EU standard size
% \usepackage[letterpaper,top=2cm,bottom=2cm,left=3cm,right=3cm,marginparwidth=1.75cm]{geometry}

% Useful packages
\usepackage[margin=2.54cm]{geometry}
\usepackage{amsmath}
\usepackage{amsfonts}
\usepackage{graphicx}
\usepackage{hyperref}
\usepackage{multirow}
\usepackage{csquotes}
\usepackage{mathtools}
\DeclarePairedDelimiter\ceil{\lceil}{\rceil}
\DeclarePairedDelimiter\floor{\lfloor}{\rfloor}
\newcommand{\ie}{\textit{i}.\textit{e}.}
\title{VE475 Homework 2}
\author{Yiwen Yang 519370910053}

	
\begin{document}
\date{}
\maketitle

\section*{Ex.1 - Simple Questions}

\begin{enumerate}
	\item 
	\begin{align*}
		101 &= 5 \times 17 + 16\\
		17 &= 1 \times 16 + 1\\
		16 &= 16 \times 1\\
		1 &= 17 - 1 \times 16\\
		&= 6 \times 17 - 101
	\end{align*}
	So the inverse of 17 modulo 101 is 6.
	\item To solve $12x \equiv 28 \bmod 236$, is equivalent to solve $12x + 236y = 28 \Leftrightarrow 3x + 59y = 7$.
	\begin{align*}
		59 &= 19 \times 3 + 2\\
		3 &= 1 \times 2 + 1\\
		2 &= 2 \times 1\\
		1 &= 3 - 1 \times 2\\
		&= 20 \times 3 - 59\\
		7 &= 140 \times 3 + (-7) \times 59
	\end{align*}
	So $x = 140 + 59t$, $t \in \mathbb{Z}$.
	\item
	From Euclidean algorithm, one can infer that
	\begin{displaymath}
		\left\{
		\begin{array}{l}
			gcd(c, 31) = 1\\
			gcd(7, \phi(31)) = gcd(7, 30) = 1
		\end{array}
		\right.
	\end{displaymath}
	Using extended Euclidean algorithm to find $p$ such that $7p \equiv 1 \bmod 30$, one can get $p = 13 + 30t$, $t \in \mathbb{Z}$. Assume $t = 0$, then the plain text $m \equiv c^p \equiv c^{13} \bmod 31$.
	\item
	\begin{align*}
		4883 &= 19 \times 257\\
		4369 &= 17 \times 257
	\end{align*}
	\item Let 
		$M = 
		\begin{pmatrix}
			3 & 5\\
			7 & 3
		\end{pmatrix}$. 

	$M$ is not invertible modulo $p$ implies that $det(M \bmod p) = 0$.

	When $p > 7$, $det(M \bmod p) = det(M) = -26$, $M$ is invertible.

	Try all $p \le 7$: $p = 2, 3, 5, 7$, one can get that $det(M \bmod 2) = 0$. So when $p = 2$, $M$ is not invertible.
	\item Suppose $a$ and $p$ are coprime, \ie, $gcd(a, p) = 1$. Then according to Bézout's identity, $\exists r, s \in \mathbb{Z}$ such that $ra + sp = 1$. Multiply both sides by $b$, one can get $rab + spb = b$. Since $p \mid rab, p \mid spb$, then $p \mid b$, \ie, $b \equiv 0 \bmod p$.

	Similarly, if $b$ and $p$ are coprime, then $a \equiv 0 \bmod p$. So at least one of $a, b$ should be divisible by $p$.
	\item 
	\begin{align*}
		2^{2017} &\equiv (2^2)^{1008} \times 2\\
		&\equiv (-1)^{1008} \times 2\\
		&\equiv 2 \bmod 5
	\end{align*}
	\begin{align*}
		2^{2017} &\equiv (2^6)^{336} \times 2\\
		&\equiv (-1)^{336} \times 2\\
		&\equiv 2 \bmod 13
	\end{align*}
	\begin{align*}
		2^{2017} &\equiv (2^5)^{403} \times 4\\
		&\equiv 1^{403} \times 4\\
		&\equiv 4 \bmod 31
	\end{align*}
	Since $2015 = 5 \times 13 \times 31$, Chinese Remainder Theorem may be used to calculated $2^{2017} \bmod 2015$.
	\begin{align*}
		(-161) \times 5 + 2 \times 403 &= 1\\
		12 \times 13 + (-1) \times 155 &= 1\\
		21 \times 31 + (-10) \times 65 &= 1
	\end{align*}
	\begin{align*}
		2^{2017} &\equiv 2 \times 2 \times 403 + 2 \times (-1) \times 155 + 4 \times (-10) \times 65\\
		&\equiv -1298\\
		&\equiv 717 \bmod 2015
	\end{align*}
\end{enumerate}

\section*{Ex.2 - Rabin Cryptosystem}

\begin{enumerate}
	\item Find two big prime number $p$ and $q$, and $p, q \equiv 3 \bmod 4$. Rabin Cryptosystem takes $n = pq$ as public key, $(p, q)$ as private key. 

	To encrypt plain text $m$, cipher text $c$ is calculated as 
	\begin{align*}
		c = m^2 \bmod n
	\end{align*}
	To decrypt cipher text, solve the simultaneous congruence equations
	\begin{displaymath}
		\left\{
		\begin{array}{l}
			m_p \equiv c^{\frac{p + 1}{4}} \bmod p\\
			m_q \equiv c^{\frac{q + 1}{4}} \bmod q
		\end{array}
		\right.
	\end{displaymath}
	Where $m_p$ and $m_q$ are defined as
	\begin{displaymath}
		\left\{
		\begin{array}{l}
			m \equiv \pm m_p \bmod p\\
			m \equiv \pm m_q \bmod q
		\end{array}
		\right.
	\end{displaymath}
	Apply the Chinese Remainder Theorem, one can get four different solutions for the plain text.
	\item 
	\begin{enumerate}
		\item Since at most four different solutions can be found, there is $25\%$ of chance to observe a meaningful message when feeding random numbers.
		\item No, since if Eve only has $x$ and public key $n$, he cannot solve like above. Whether to factorize $n$, or to solve $m \equiv \sqrt{x} \bmod n$ are as the same difficulty.
		\item Eve should run a chosen cipher attack. Eve can then use the device to get the four different solutions as $\pm a, \pm b$. Since $gcd(|a - b|, n)$ is a non-trivial factor of $n$, then Eve can factorize $n$ by calculating $|a - b|$, which is either $p$ or $q$.
	\end{enumerate}
\end{enumerate}

\section*{Ex.3 - CRT}

Assume there are $x$ people in the group, then $x$ satisfies that
\begin{align*}
	x &\equiv 1 \bmod 3\\
	x &\equiv 2 \bmod 4\\
	x &\equiv 3 \bmod 5
\end{align*}
Since $3,4,5$ are coprime to each other, one can use Chinese Remainder Theorem to solve $x$.
\begin{align*}
	7 \times 3 + (-1) \times 20 &= 1\\
	4 \times 4 + (-1) \times 15 &= 1\\
	5 \times 5 + (-2) \times 12 &= 1
\end{align*}
\begin{align*}
	x &\equiv 1 \times (-1) \times 20 + 2 \times (-1) \times 15 + 3 \times (-2) \times 12\\
	&\equiv -122\\
	&\equiv 58 \bmod 60
\end{align*}
Then the two smallest possible $x$ are 58 and 118.

\end{document}