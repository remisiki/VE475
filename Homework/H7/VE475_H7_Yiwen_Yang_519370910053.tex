\documentclass[12pt]{article}

% Language setting
% Replace `english' with e.g. `spanish' to change the document language
\usepackage[english]{babel}

% Set page size and margins
% Replace `letterpaper' with`a4paper' for UK/EU standard size
% \usepackage[letterpaper,top=2cm,bottom=2cm,left=3cm,right=3cm,marginparwidth=1.75cm]{geometry}

% Useful packages
\usepackage[margin=2.54cm]{geometry}
\usepackage[section]{placeins}
\usepackage{algorithm}
\usepackage{algpseudocode}
\usepackage{amsfonts}
\usepackage{amsmath}
\usepackage{amssymb}
\usepackage{caption}
\usepackage{csquotes}
\usepackage{enumerate}
\usepackage{enumitem}
\usepackage{float}
\usepackage{graphicx}
\usepackage{hyperref}
\usepackage{indentfirst}
\usepackage{multirow}
\usepackage{subcaption}
\title{VE475 Homework 7}
\author{Yiwen Yang}


\begin{document}
\date{}
\maketitle

\section*{Ex. 0 - Merkle-Damgård Construction}

	\begin{enumerate}
		\item
			\begin{enumerate}
				\item
					For any $y$, we can derive $x$ by checking from the tail: if last bit is 0, then $x_i=0$; else last two bits must be 01, then $x_i=1$. The result of $x$ is unique, so there is no different $x'$ that can produce the same $y$, then $s$ is injective.
				\item
					If $z$ is empty, then from (a) we know that no $s(x)=s(x')$. If $z$ is not empty, since all $s(x')$ starts with 11, and we cannot find two adjacent 1 of $s(x)$ which consists of only 0 and 01 (except at the beginning), so $s(x)$ cannot be $z\mathbin\Vert s(x')$.
				\end{enumerate}
		\item
			(a) is important because if $s$ is not injective, then we can found $x\neq x'$ such that $y=y'$, which contributes a collision to $h$. (b) is important because if the condition is true, then we can pad $z$ in front of $s(x')$ to find another $s(x)$; When $z'$ is empty, this means $g(z'\mathbin\Vert s(x))=g(z\mathbin\Vert s(x'))$ and this is a collision.
		\item
			Assume there is a collision on $h$, $x\neq x'$ such that $h(x)=h(x')$, we will prove that a collision on the compression function $g$ can be efficiently found.

			First, if $x$ and $x'$ has the same block numbers, i.e. $k=k'$, then $y_k=y_{k'}$.
			$$g(z_{k-1}\mathbin\Vert y_k)=g(z_{k'-1}\mathbin\Vert y_{k'})$$

			If $z_{k-1}\neq z_{k'-1}$, then collision if found. Otherwise repeat the process, then either we find a collision or at last $z_1=z_{1'}\ldots z_k=z_{k'}$, which contradicts.

			Next, if $x'$ is longer than $x$, i.e. $k'>k$, repeat the same process as above and either find a collision or at last but not least
			$$g(z_1\mathbin\Vert y_2)=g(z_{k'-k}\mathbin\Vert y_{k'-k+1})$$
			Since $y_2=1$, if $y_{k'-k+1}=0$ then collision if found. Else $y_{k'-k+1}=1$, $y_{k'-k}=0$ because there is no continuous two 1 in $y$ except at the beginning. Then
			$$z_1=z_{k'-k}$$
			$$g(0^m\mathbin\Vert y_1)=g(z_{k'-k-1}\mathbin\Vert y_{k'-k})$$
			Then collision is found since $y_1=1$ while $y_{k'-k}=0$.

			Above all cases, $h$ should be collision resistant if $h$ is collision resistant.
	\end{enumerate}

\section*{Ex. 1 - Cramer-Shoup Cryptosystem}

	\begin{enumerate}
		\item
			\begin{itemize}
				\item
					\textbf{Key Generation}

					Given a cyclic group $G$ of order $q$, find two random different generators $g_1$ and $g_2$. First, choose $(x_1,x_2,y_1,y_2,z)$ from $\{0,\ldots,q-1\}$ randomly. Then computes $c=g_1^{x_1}g_2^{x_2}$, $d=g_1^{y_1}g_2^{y_2}$, $h=g_1^z$. Finally, $(G,q,g_1,g_2,c,d,h)$ are public keys, and $(x_1,x_2,y_1,y_2,z)$ are private keys.
				\item
					\textbf{Encryption}

					First choose a random $k$ from $\{0,\ldots,q-1\}$. Then computes $u_1=g_1^k$, $u_2=g_2^k$, $e=h^km$, $\alpha=H(u_1,u_2,e)$, $v=c^kd^{k\alpha}$, where $H()$ is a universal one-way hash function. The cipher text is given as $(u_1,u_2,e,v)$.
				\item
					\textbf{Decryption}

					First computes $\alpha=H(u_1,u_2,e)$ with the same hash function and verifies that $v=u_1^{x_1}u_2^{x_2}(u_1^{y_1}u_2^{y_2})^{\alpha}$. If fails, then reject the decryption. Otherwise, the plain text is given by $m=e/(u_1^z)$.
			\end{itemize}
		\item
			This cryptosystem is IND-CCA2 secure because there is a verification step that checks whether the hash function gives the correct result, otherwise the whole encryption is rejected. This can prevent the attacker trying to discover some patterns when feeding the oracle with some invalid or forged ciphertexts.
		\item
			Cramer-Shoup and Elgamal are both public key encryption algorithms based on DLP, while Cramer-Shoup is based on the computational intractability of DDH and Elgamal is based on both CDH and DDH. Elgamal is not IDN-CCA safe, while Cramer-Shoup is.
	\end{enumerate}

\section*{Ex. 2 - Simple Questions}

	\begin{enumerate}
		\item
			For any given $x$, we can easily find $x'=x+p-1$ such that $h(x)\equiv \alpha^x\equiv\alpha^{x+p-1}\equiv\alpha^{x'}\equiv h(x')\bmod p$, so $h$ is neither collision resistant nor second pre-image resistant.
		\item
			The results are 0x5a827999, 0x6ed9eba1, 0x8f1bbcdc and 0xca62c1d6, which are equal to the four constants in $K$.
	\end{enumerate}

\section*{Ex. 3 - Birthday Paradox}

	\begin{enumerate}
		\item
			\begin{enumerate}
				\item
					$$g(x)=\ln(1-x)+x+x^2$$
					$$\frac{d}{dx}g(x)=\frac{1}{x-1}+1+2x=\frac{2(x-\frac{1}{4})^2-\frac{1}{8}}{x-1}$$
					When $x\in[0,\frac{1}{2}]$, $\frac{d}{dx}g(x)>0$, so $g(x)$ is monotonically increasing. Note that $g(0)=0$, then $g(x)\geq0\Rightarrow-x-x^2\leq\ln(1-x)$.
				\item
					$$h(x)=\ln(1-x)+x$$
					$$\frac{d}{dx}h(x)=\frac{1}{x-1}+1=\frac{x}{x-1}$$
					When $x\in[0,\frac{1}{2}]$, $\frac{d}{dx}h(x)<0$, so $h(x)$ is monotonically decreasing. Note that $h(0)=0$, then $h(x)\leq0\Rightarrow\ln(1-x)\leq-x$.
			\end{enumerate}
			From above, $-x-x^2\leq\ln(1-x)\leq-x$.
		\item
			Since $r\leq n/2$, then $r/n\leq1/2$. From (1) we can get
			$$-\frac{j}{n}-\frac{j^2}{n^2}\leq\ln(1-\frac{j}{n})\leq-\frac{j}{n}$$
			$$\sum_{j=1}^{r-1}(-\frac{j}{n}-\frac{j^2}{n^2})\leq\sum_{j=1}^{r-1}\ln(1-\frac{j}{n})\leq\sum_{j=1}^{r-1}-\frac{j}{n}$$
			$$-\frac{(r-1)r}{2n}-\frac{(r-1)r(2r-1)}{6n^2}\leq\sum_{j=1}^{r-1}\ln(1-\frac{j}{n})\leq-\frac{(r-1)r}{2n}$$
			On the left hand side
			$$\frac{(r-1)r(2r-1)}{6n^2}=\frac{r^3-\frac{3}{2}r^2+r}{3n^2}<\frac{r^3}{3n^2}$$
			So
			$$-\frac{(r-1)r}{2n}-\frac{r^3}{3n^2}\leq\sum_{j=1}^{r-1}\ln(1-\frac{j}{n})\leq-\frac{(r-1)r}{2n}$$
		\item
			Let $\lambda=r^2/2n$, and $\lambda\leq n/8\Rightarrow r\leq n/2$, then
			$$-\frac{(r-1)r}{2n}-\frac{r^3}{3n^2}=-\lambda+\sqrt{\frac{\lambda}{2n}}-\frac{(2\lambda)^{3/2}}{3n^{1/2}}=-\lambda+c_1/\sqrt{n}$$
			$$-\frac{(r-1)r}{2n}=-\lambda+\sqrt{\frac{\lambda}{2n}}=-\lambda+c_2/\sqrt{n}$$
			Then take the exponential of (2) gets
			$$e^{-\lambda+c_1/\sqrt{n}}\leq\prod_{j=1}^{r-1}(1-\frac{j}{n})\leq e^{-\lambda+c_2/\sqrt{n}}$$
		\item
			If $\lambda$ is a constant, then $c_1$ and $c_2$ from (3) are all constants, which means that when $n$ is very large, $e^{c_1/\sqrt{n}},e^{c_2/\sqrt{n}}\rightarrow1$. So $\prod_{j=1}^{r-1}(1-\frac{j}{n})\fallingdotseq e^{-\lambda}$.
	\end{enumerate}

\section*{Ex. 4 - Birthday Attack}

	\begin{enumerate}
		\item
			$$1-\prod_{i=1}^{39}\frac{1000-i}{1000}=0.546$$
			There is 0.546 probability to observe two plates with the same 3-digit ending.
		\item
			$$C_{40}^1\frac{1}{1000}(\frac{999}{1000})^{39}=0.038$$
			There is 0.038 probability to find one plate with 123 ending.
		\item
			The probability of having two same birthday in a global range is very high, but if we select a certain birthday, then the collision rate will be much smaller. Similarly, Alice first chooses a hash to sign on, then the hash collision rate will be much lower than that of arbitrary two hash which Eve provides.
	\end{enumerate}

\section*{Ex. 5 - Faster Multiple Modular Exponentiation}

	\begin{enumerate}
		\item
			$\mathcal{O}((\log{n})^2\log{a}+(\log{n})^2\log{b})=\mathcal{O}((\log{n})^2\log{ab})$
		\item
			If $\alpha\beta$ is given, then the following algorithm can be used.
			\begin{algorithm}[H]
				\caption{Faster multiple modular exponentiation}
				\textbf{Input:} $\alpha$, $a$, $\beta$, $b$, $n$, $\alpha\beta$\\
				\textbf{Output:} $\alpha^a\beta^b\bmod n$
				\begin{algorithmic}
					\State $k\gets$ Max(Bit size of $a$, Bit size of $b$)
					\State $power \gets 1$
					\For{$i \gets k-1$ downto 0}
						\State $power \gets power \cdot power \bmod n$
						\If{$a_i=1$ \textbf{and} $b_i=1$}
							\State $power \gets power \cdot \alpha\beta \bmod n$
						\ElsIf{$a_i=1$}
							\State $power \gets power \cdot \alpha \bmod n$
						\ElsIf{$b_i=1$}
							\State $power \gets power \cdot \beta \bmod n$
						\EndIf
					\EndFor
					\State \Return{$power$}
				\end{algorithmic}
			\end{algorithm}
		\item
			$l$ squaring and $l$ multiplications are necessary.
		\item
			See source code in folder \textbf{ex5}. The results show that when calculating random 15360-bit numbers, the normal method takes 1.403902s and faster method takes 0.952246s (32\% faster).
	\end{enumerate}

\end{document}