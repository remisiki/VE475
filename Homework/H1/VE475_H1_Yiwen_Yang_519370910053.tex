\documentclass[12pt]{article}

% Language setting
% Replace `english' with e.g. `spanish' to change the document language
\usepackage[english]{babel}

% Set page size and margins
% Replace `letterpaper' with`a4paper' for UK/EU standard size
% \usepackage[letterpaper,top=2cm,bottom=2cm,left=3cm,right=3cm,marginparwidth=1.75cm]{geometry}

% Useful packages
\usepackage[margin=2.54cm]{geometry}
\usepackage{amsmath}
\usepackage{amsfonts}
\usepackage{graphicx}
\usepackage{hyperref}
\usepackage{multirow}
\usepackage{csquotes}
\usepackage{mathtools}
\DeclarePairedDelimiter\ceil{\lceil}{\rceil}
\DeclarePairedDelimiter\floor{\lfloor}{\rfloor}
\title{VE475 Homework 1}
\author{Yiwen Yang 519370910053}

	
\begin{document}
\date{}
\maketitle

\section*{Ex.1 - Simple Questions}

\begin{enumerate}
	\item Use brute force to list all possible plain text with the key $\kappa$ from 0 to 25, the results are shown as:

	EVIRE, DUHQD, CTGPC, BSFOB, ARENA, ZQDMZ, YPCLY, XOBKX, WNAJW, VMZIV, ULYHU, TKXGT, SJWFS, RIVER, QHUDQ, PGTCP, OFSBO, NERAN, MDQZM, LCPYL, KBOXK, JANWJ, IZMVI, HYLUH, GXKTG, FWJSF

	Then through observation, the most possible plain text is RIVER, so Bob should go to the river to meet Alice.

	\item 
	\begin{enumerate}
		\item Text {\it dont} consists of 4 characters, so block size $n$ should satisfy that $n \mid 4$. Try $n=2$.
		\item The order of {\it dont} in alphabet is $[3,14,13,9]$, and ELNI is $[4,11,13,8]$. 

		Construct the equation as
		\begin{align*}
			A &=
			\begin{pmatrix}
				3 & 14\\
				13 & 9\\
			\end{pmatrix}
			\\
			\begin{pmatrix}
				3 & 14\\
				13 & 9\\
			\end{pmatrix}
			\cdot
			\begin{pmatrix}
				a & b\\
				c & d\\
			\end{pmatrix}
			&\equiv
			\begin{pmatrix}
				4 & 11\\
				13 & 8\\
			\end{pmatrix}
			\bmod 26
		\end{align*}

		\item Since $det(A)=-125$, $gcd(-125, 26)=1$, then $A$ is invertible modulo 26. 

		Calculate the inverse of $-125$ modulo $26$
		\begin{align*}
			-125 &= -5 \times 26 + 5\\
			26 &= 5 \times 5 + 1\\
			1 &= 26 - 5 \times 5\\
			&= (-5) \times (-125) + (-24) \times 26\\
			(-125)^{-1} &= (-5)
		\end{align*}

		The inverse of $A$ is
		\begin{align*}
			A^{-1} &=
			-\frac{1}{125}
			\begin{pmatrix}
				19 & -14\\
				-13 & 3\\
			\end{pmatrix}
			\\
			&=
			\begin{pmatrix}
				-95 & 70\\
				65 & -15\\
			\end{pmatrix}
			\\
			&\equiv
			\begin{pmatrix}
				9 & 18\\
				13 & 11\\
			\end{pmatrix}
			\bmod 26
		\end{align*}

		Calculate the key as
		\begin{align*}
			K = 
			\begin{pmatrix}
				a & b\\
				c & d\\
			\end{pmatrix}
			&=
			A^{-1} \cdot 
			\begin{pmatrix}
				4 & 11\\
				13 & 8\\
			\end{pmatrix}
			\\
			&=
			\begin{pmatrix}
				9 & 18\\
				13 & 11\\
			\end{pmatrix}
			\cdot
			\begin{pmatrix}
				4 & 11\\
				13 & 8\\
			\end{pmatrix}
			\\
			&=
			\begin{pmatrix}
				270 & 243\\
				195 & 231\\
			\end{pmatrix}
			\\
			&\equiv
			\begin{pmatrix}
				10 & 9\\
				13 & 23\\
			\end{pmatrix}
			\bmod 26
		\end{align*}

		So the encryption matrix is
		$
		\begin{pmatrix}
			10 & 9\\
			13 & 23\\
		\end{pmatrix}
		$.
	\end{enumerate}

	\item Since $n \mid ab$, then $\exists m \in \mathbb{Z}^{*}$ such that $ab=mn$.

	From $gcd(a, n)=1$, one can find $x, y \in \mathbb{Z}^{*}$ such that $ax + ny = 1$.

	Then
	\begin{align*}
		b &= b(ax+ny)\\
		&= abx+bny\\
		&= mnx+bny\\
		&= n(mx+by)\\
	\end{align*}
	Since $m,x,b,y$ are all integers, then $n \mid b$.

	\item 
	\begin{enumerate}
		\item 
		\begin{align*}
			30030 &= 116 \times 257 + 218\\
			257 &= 1 \times 218 + 39\\
			218 &= 5 \times 39 + 23\\
			39 &= 1 \times 23 + 16\\
			23 &= 1 \times 16 + 7\\
			16 &= 2 \times 7 + 2\\
			7 &= 3 \times 2 + 1\\
			2 &= 2 \times 1 + 0
		\end{align*}
		Thus $gcd(30030, 257)=1$.
		\item Since $\floor*{\sqrt{257}}=16$, check prime numbers less than 16 which are $[2,3,5,7,11,13]$ and non of the numbers is a divisor of 257, prooving that 257 is prime.
	\end{enumerate}

	\item Since the key of One Time Pad is very easy to calculate given plain text and cipher text, so it is very likely for Eve to observe a second time usage of the key in a known plaintext attack, which is not safe.
	\item Since secure means that the attacker has to compute at least $2^{128}$ operations to break the encryption, 
	\begin{align*}
		\sqrt{n \log{n}} &\ge 128 \\
		n &\ge 4487 \\
	\end{align*}
	Hence the size of the graph should be at least 4487.
\end{enumerate}

\section*{Ex.2 - Vigenère Cipher}

\begin{enumerate}
	\item Vigenère cipher is designed based on Caesar shift cipher. In Caesar cipher, every plain text character is mapped to the cipher text character by adding a Caesar shift $n$, where $0 \le n \le 25$. To encrypt a Vigenère cipher, first define a key $K$ with length $l$. For each character in plain text with index $i$, find the corresponding shift key in $K$ as $k_i=K_{i \bmod l}$ (${K_p}$ is the $p$th character of $K$). Then, construct a Caesar shift $n_i$ on the plain text character, where $n_i$ is the shifting number of plain text $A$ to Caesar cipher $k_i$. Apply to all of the characters in plain text, one will get a Vigenère cipher. To decrypt the cipher, simply find all corresponding Caesar unshift with $K$.

	Vigenère cipher utilizes multiple Caesar ciphers in a loop of the key length to achieve the encryption, and the reference table is shown in Table \ref{table:vigenere}.

	For example, to encrypt plain text TORADORA with key TAIGA, find corresponding Caesar shift for the key as
	\begin{align*}
		T &\rightarrow \textnormal{Caesar shift 19} \\
		A &\rightarrow \textnormal{Caesar shift 0} \\
		I &\rightarrow \textnormal{Caesar shift 8} \\
		G &\rightarrow \textnormal{Caesar shift 6} \\
		A &\rightarrow \textnormal{Caesar shift 0} \\
	\end{align*}
	Then repeating the 5 Caesar shift on the plain text TORADORA yields
	\begin{align*}
		T &\rightarrow \textnormal{Caesar shift 19} \rightarrow M \\
		O &\rightarrow \textnormal{Caesar shift 0} \rightarrow O \\
		R &\rightarrow \textnormal{Caesar shift 8} \rightarrow Z \\
		A &\rightarrow \textnormal{Caesar shift 6} \rightarrow G \\
		D &\rightarrow \textnormal{Caesar shift 0} \rightarrow D \\
		O &\rightarrow \textnormal{Caesar shift 19} \rightarrow H \\
		R &\rightarrow \textnormal{Caesar shift 0} \rightarrow R \\
		A &\rightarrow \textnormal{Caesar shift 8} \rightarrow I \\
	\end{align*}
	Then the Vigenère cipher for this example is MOZGDHRI. It can be observed that same characters in the plain text such as $A$, have very different ciphers as $G$ and $I$. This makes the word frequency attack on Caesar cipher become extremely difficult.
	\begin{table}[h]
		\centering
		\resizebox{0.9\textwidth}{!}{
			\begin{tabular}{|c|c|c|c|c|c|c|c|c|c|c|c|c|c|c|c|c|c|c|c|c|c|c|c|c|c|c|}
				\hline
				 & A & B & C & D & E & F & G & H & I & J & K & L & M & N & O & P & Q & R & S & T & U & V & W & X & Y & Z\\
				\hline
				0 & A & B & C & D & E & F & G & H & I & J & K & L & M & N & O & P & Q & R & S & T & U & V & W & X & Y & Z\\
				\hline
				1 & B & C & D & E & F & G & H & I & J & K & L & M & N & O & P & Q & R & S & T & U & V & W & X & Y & Z & A\\
				\hline
				2 & C & D & E & F & G & H & I & J & K & L & M & N & O & P & Q & R & S & T & U & V & W & X & Y & Z & A & B\\
				\hline
				3 & D & E & F & G & H & I & J & K & L & M & N & O & P & Q & R & S & T & U & V & W & X & Y & Z & A & B & C\\
				\hline
				4 & E & F & G & H & I & J & K & L & M & N & O & P & Q & R & S & T & U & V & W & X & Y & Z & A & B & C & D\\
				\hline
				5 & F & G & H & I & J & K & L & M & N & O & P & Q & R & S & T & U & V & W & X & Y & Z & A & B & C & D & E\\
				\hline
				6 & G & H & I & J & K & L & M & N & O & P & Q & R & S & T & U & V & W & X & Y & Z & A & B & C & D & E & F\\
				\hline
				7 & H & I & J & K & L & M & N & O & P & Q & R & S & T & U & V & W & X & Y & Z & A & B & C & D & E & F & G\\
				\hline
				8 & I & J & K & L & M & N & O & P & Q & R & S & T & U & V & W & X & Y & Z & A & B & C & D & E & F & G & H\\
				\hline
				9 & J & K & L & M & N & O & P & Q & R & S & T & U & V & W & X & Y & Z & A & B & C & D & E & F & G & H & I\\
				\hline
				10 & K & L & M & N & O & P & Q & R & S & T & U & V & W & X & Y & Z & A & B & C & D & E & F & G & H & I & J\\
				\hline
				11 & L & M & N & O & P & Q & R & S & T & U & V & W & X & Y & Z & A & B & C & D & E & F & G & H & I & J & K\\
				\hline
				12 & M & N & O & P & Q & R & S & T & U & V & W & X & Y & Z & A & B & C & D & E & F & G & H & I & J & K & L\\
				\hline
				13 & N & O & P & Q & R & S & T & U & V & W & X & Y & Z & A & B & C & D & E & F & G & H & I & J & K & L & M\\
				\hline
				14 & O & P & Q & R & S & T & U & V & W & X & Y & Z & A & B & C & D & E & F & G & H & I & J & K & L & M & N\\
				\hline
				15 & P & Q & R & S & T & U & V & W & X & Y & Z & A & B & C & D & E & F & G & H & I & J & K & L & M & N & O\\
				\hline
				16 & Q & R & S & T & U & V & W & X & Y & Z & A & B & C & D & E & F & G & H & I & J & K & L & M & N & O & P\\
				\hline
				17 & R & S & T & U & V & W & X & Y & Z & A & B & C & D & E & F & G & H & I & J & K & L & M & N & O & P & Q\\
				\hline
				18 & S & T & U & V & W & X & Y & Z & A & B & C & D & E & F & G & H & I & J & K & L & M & N & O & P & Q & R\\
				\hline
				19 & T & U & V & W & X & Y & Z & A & B & C & D & E & F & G & H & I & J & K & L & M & N & O & P & Q & R & S\\
				\hline
				20 & U & V & W & X & Y & Z & A & B & C & D & E & F & G & H & I & J & K & L & M & N & O & P & Q & R & S & T\\
				\hline
				21 & V & W & X & Y & Z & A & B & C & D & E & F & G & H & I & J & K & L & M & N & O & P & Q & R & S & T & U\\
				\hline
				22 & W & X & Y & Z & A & B & C & D & E & F & G & H & I & J & K & L & M & N & O & P & Q & R & S & T & U & V\\
				\hline
				23 & X & Y & Z & A & B & C & D & E & F & G & H & I & J & K & L & M & N & O & P & Q & R & S & T & U & V & W\\
				\hline
				24 & Y & Z & A & B & C & D & E & F & G & H & I & J & K & L & M & N & O & P & Q & R & S & T & U & V & W & X\\
				\hline
				25 & Z & A & B & C & D & E & F & G & H & I & J & K & L & M & N & O & P & Q & R & S & T & U & V & W & X & Y\\
				\hline

			\end{tabular}
		}
		\caption{Vigenère Cipher}
		\label{table:vigenere}
	\end{table}
	\item 
		\begin{enumerate}
			\item When Eve has the copy of the cipher text, he will find that the cipher repeats itself at a certain length (in this example 6), indicating that the Bob is sending the same letter.
			\item Observe the cipher and find the least cycle period, then the period is the key length.
			\item Extract any 6 letters from the cipher text, apply Caesar unshift $n$ ($n=0,1,\ldots ,25$) on the cipher text to get 26 possible shifted keys. Since no English word with 6 letters is a shift of another one, observe all these 26 keys and their shifted version to find an English word, then the word will be the key.
		\end{enumerate}
\end{enumerate}

\section*{Ex.3 - Programming}

\paragraph{}

See in folder {\bf ex3}.

\end{document}