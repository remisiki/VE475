\documentclass[12pt]{article}

% Language setting
% Replace `english' with e.g. `spanish' to change the document language
\usepackage[english]{babel}

% Set page size and margins
% Replace `letterpaper' with`a4paper' for UK/EU standard size
% \usepackage[letterpaper,top=2cm,bottom=2cm,left=3cm,right=3cm,marginparwidth=1.75cm]{geometry}

% Useful packages
\usepackage[margin=2.54cm]{geometry}
\usepackage{amsmath}
\usepackage{amsfonts}
\usepackage{graphicx}
\usepackage{hyperref}
\usepackage{multirow}
\usepackage{csquotes}
\usepackage[section]{placeins}
\usepackage{caption}
\usepackage{subcaption}
\usepackage{indentfirst}
\usepackage{enumerate}
\title{VE475 Homework 3}
\author{Yiwen Yang}

	
\begin{document}
\date{}
\maketitle

\section*{Ex.1 - Finite Fields}

\begin{enumerate}
	\item
		$X^2+1$ is of degree 2, so all possible factors should be of degree 1, which are $X$, $X + 1$, $X + 2$ in $\mathbb{F}_3[X]$.

		Try
		\begin{align*}
			X \cdot X&=X^2\\
			(X+1)(X+1)&=X^2+2X+1\\
			(X+2)(X+2)&=X^2+4X+4=X^2+X+1\\
			X(X+1)&=X^2+X\\
			X(X+2)&=X^2+2X\\
			(X+1)(X+2)&=X^2+3X+2=X^2+2
		\end{align*}

		None of the above polynomials equal to $X^2+1$, so $X^2+1$ is irreducible.
	\item
		Let $P(X)=X^2+1$, $A(X)=1+2X$, and there exists another two different $B(X)$ and $C(X)$ such that
		\begin{align*}
			A(X)B(X) \equiv A(X)C(X) \bmod P(X)
		\end{align*}

		Then $A(X)(B(X)-C(X)) \equiv 0 \bmod P(X)$, but $P(X)$ is irreducible in $\mathbb{F}_3[X]$, so it contradicts. This implies that there exists $B(X)$ such that 
		\begin{align*}
			A(X)B(X) \equiv 1 \bmod P(X)
		\end{align*}

		And $B(X)$ is the multiplicative inverse of $A(X)$ modulo $P(X)$. So multiplicative inverse of $1+2X$ modulo $X^2+1$ exists in $\mathbb{F}_3[X]$.
	\item
		\begin{align*}
			X^2+1 &= (1+2X)2X+(-2X+1)\\
			1+2X &= (-2X+1)(-1)+2\\
			-2X+1 &= 2(-X)+1\\
			1 &= (-2X+1)+2X\\
			&= (-2X+1)+((1+2X)+(-2X+1))X\\
			&= (1+2X)X+(X+1)(-2X+1)\\
			&= (1+2X)X+(X+1)((X^2+1)-(1+2X)2X)\\
			&= (X+1)(X^2+1)+(-2X^2-X)(1+2X)\\
			&= (X+1)(X^2+1)+(X^2+2X)(1+2X)
		\end{align*}

		So the multiplicative inverse of $1+2X$ modulo $X^2+1$ in $\mathbb{F}_3[X]$ is $X^2+2X$.
\end{enumerate}

\section*{Ex.2 - AES}

\begin{enumerate}
	\item
		\begin{enumerate}
			\item
				Cyclically shift to the right row $i$ by offset $i$, $0 \le i \le 3$.
			\item
				The same as the encryption process, xor each byte with a byte of round key.
			\item
				The MixColumns Layer is calculated by $C(X) \cdot X$, then to reverse it simply calculate $C^{-1}(X)$ multiplies the matrix. Multiply $C(X)$ by the given InvMixColumns yields 1, so InvMixColumns is $C^{-1}(X)$.
		\end{enumerate}
	\item
		First AddRoundKey to the cipher text, then repeat (InvShiftRows, InvSubBytes, AddRoundKey, InvMixColumns) for 9 round, and for round 10, apply (InvShiftRows, InvSubBytes, AddRoundKey) to get the plain text.
	\item
		Since InvShiftRows simply shifts the row position without changing values, InvSubBytes substitutes values without changing positions, the two steps may be reversed without changing the result.
	\item
		\begin{enumerate}
			\item
				AddRoundKey and InvMixColumns both overwrites the cell value of the matrix, so reversing the two processes may lead to different results.
			\item
				\begin{align*}
					e_{i,j}=(m_{i,j}a_{i,j})\oplus k_{i,j}
				\end{align*}
			\item
				\begin{align*}
					m_{i,j}a_{i,j}&=e_{i,j}\oplus k_{i,j}\\
					a_{i,j}&=m_{i,j}^{-1}(e_{i,j}\oplus k_{i,j})\\
					&=(m_{i,j}^{-1}e_{i,j})\oplus(m_{i,j}^{-1}k_{i,j})
				\end{align*}
			\item
				When generating inverse round keys, left multiply the round keys with the inverse mix columns, \textit{i.e.}, $k^{'}_{i,j}=m_{i,j}^{-1}k_{i,j}$. Then InvAddRoundKey simply xor the matrix generated from InvMixColumns with the inverse round keys.
		\end{enumerate}
	\item
		First AddRoundKey to the cipher text, then repeat (InvShiftRows, InvSubBytes, InvMixColumns, InvAddRoundKey) for 9 round, and for round 10, apply (InvShiftRows, InvSubBytes, AddRoundKey) to get the plain text.
	\item
		The decryption process is as the same structure of the encryption process, only needed to implemented the inverse functions to replace into the layers without fixing the layer orders.
\end{enumerate}

\section*{Ex.3 - DES}

\begin{enumerate}
	\item
		DES encrypts and decrypts block cipher based on Feistel Network. Encryption uses a 64-bit key and the block size as 64-bit, decryption uses the same key and the cipher text block is also 64-bit. The key is 64-bit in length, but every 8-bit there is a odd parity bit, \textit{i.e.}, the key is composed of a 56-bit key and 8 parity bits. The 64-bit is processed by key schedule (KS) to generate 16 sub keys $K_1 \sim K_{16}$, each of 48-bit.

		The general process of DES goes through as
		\begin{enumerate}
			\item
				Apply Initial Permutation (IP) on the input (plain text). IP uses the Table \ref{tab:ip} to replace the bits. After IP, output should be input[58], input[50], $\ldots$, input[7].
				\begin{table}[hbtp]
					\centering
					\begin{tabular}{cccccccc}
						58 & 50 & 42 & 34 & 26 & 18 & 10 & 2\\
						60 & 52 & 44 & 36 & 28 & 20 & 12 & 4\\
						62 & 54 & 46 & 38 & 30 & 22 & 14 & 6\\
						64 & 56 & 48 & 40 & 32 & 24 & 16 & 8\\
						57 & 49 & 41 & 33 & 25 & 17 &  9 & 1\\
						59 & 51 & 43 & 35 & 27 & 19 & 11 & 3\\
						61 & 53 & 45 & 37 & 29 & 21 & 13 & 5\\
						63 & 55 & 47 & 39 & 31 & 23 & 15 & 7\\
					\end{tabular}
					\caption{Initial Permutation}
					\label{tab:ip}
				\end{table}
			\item
				\label{step:feisnet}
				Take the left 32 bits of output from IP as $L_0$, and right 32 bits as $R_0$. Feed ($L_0$, $R_0$, function $F$, sub key $K_1$) into Feistel Network, \textit{i.e.}, output ($L_1=R_0$, $R_1=L_0 \oplus F(R_0,K_1)$).
			\item
				Repeat Step \ref{step:feisnet} for 16 times. For the 16th time, switch the left and right part of output, \textit{i.e.}, ($L_{16}=L_{15} \oplus F(R_{15},K_{16})$, $R_{16}=R_{15}$).
			\item
				For the final 64-bit output of Feistel Network, apply Inverse Initial Permutation (IP-1) to get the cipher text. IP-1 is to replace bits with Table \ref{tab:ip-1}.
				\begin{table}[hbtp]
					\centering
					\begin{tabular}{cccccccc}
						40 & 8 & 48 & 16 & 56 & 24 & 64 & 32\\
						39 & 7 & 47 & 15 & 55 & 23 & 63 & 31\\
						38 & 6 & 46 & 14 & 54 & 22 & 62 & 30\\
						37 & 5 & 45 & 13 & 53 & 21 & 61 & 29\\
						36 & 4 & 44 & 12 & 52 & 20 & 60 & 28\\
						35 & 3 & 43 & 11 & 51 & 19 & 59 & 27\\
						34 & 2 & 42 & 10 & 50 & 18 & 58 & 26\\
						33 & 1 & 41 &  9 & 49 & 17 & 57 & 25\\
					\end{tabular}
					\caption{Inverse Initial Permutation}
					\label{tab:ip-1}
				\end{table}
		\end{enumerate}

		Following is detail about function $F$ and key schedule (KS).

		\begin{itemize}
			\item
				\textbf{Function $F$}

				$F$ takes 32-bit $R$ and 48-bit $K$ as input, yields a 32-bit output.
				\begin{enumerate}
					\item
						Use function $E$ to get a 48-bit output from $R$. $E$ is to replace bits by Table \ref{tab:e}.
						\begin{table}[hbtp]
							\centering
							\begin{tabular}{cccccc}
								32 &  1 &  2 &  3 &  4 &  5\\
								 4 &  5 &  6 &  7 &  8 &  9\\
								 8 &  9 & 10 & 11 & 12 & 13\\
								12 & 13 & 14 & 15 & 16 & 17\\
								16 & 17 & 18 & 19 & 20 & 21\\
								20 & 21 & 22 & 23 & 24 & 25\\
								24 & 25 & 26 & 27 & 28 & 29\\
								28 & 29 & 30 & 31 & 32 &  1\\
							\end{tabular}
							\caption{function $E$}
							\label{tab:e}
						\end{table}
					\item Xor the 48-bit result with $K$.
					\item
						Split the 48-bit result by 6 bits to get 8 partitions, and apply functions $S_1$, $S_2$, $\ldots$, $S_8$ in order. The output of each $S$ function is 4 bits, so the overall output will be 32 bits.

						Here takes $S_1$ as an example of function $S$. All 8 $S$ functions are listed in Table \ref{tab:s}.

						$S_{in}$ is the 6-bit input of $S$, use Table \ref{tab:s-1} to transform bits.

						The transform rule is defined as
						\begin{enumerate}
							\item Row index is $\{S_{in}[0]$, $S_{in}[5]\}$.
							\item Column index is $S_{in}[1:4]$.
							\item Find corresponding number (4-bit) in the table as the output.
						\end{enumerate}
						\begin{table}[hbtp]
							\centering
							\begin{tabular}{cccccccccccccccc}
								14 &  4 & 13 & 1 &  2 & 15 & 11 &  8 &  3 & 10 &  6 & 12 &  5 &  9 &  0 &  7\\
								 0 & 15 &  7 & 4 & 14 &  2 & 13 &  1 & 10 &  6 & 12 & 11 &  9 &  5 &  3 &  8\\
								 4 &  1 & 14 & 8 & 13 &  6 &  2 & 11 & 15 & 12 &  9 &  7 &  3 & 10 &  5 &  0\\
								15 & 12 &  8 & 2 &  4 &  9 &  1 &  7 &  5 & 11 &  3 & 14 & 10 &  0 &  6 & 13\\
							\end{tabular}
							\caption{S-box $S_1$}
							\label{tab:s-1}
						\end{table}
					\item
						Apply $P$ on the 32-bit output from the last step, and get the 32-bit output to be the output of function $F$. $P$ is defined to replace bits using Table \ref{tab:p}.
						\begin{table}[hbtp]
							\centering
							\begin{tabular}{cccc}
								16 &  7 & 20 & 21\\
								29 & 12 & 28 & 17\\
								 1 & 15 & 23 & 26\\
								 5 & 18 & 31 & 10\\
								 2 &  8 & 24 & 14\\
								32 & 27 &  3 &  9\\
								19 & 13 & 30 &  6\\
								22 & 11 &  4 & 25\\
							\end{tabular}
							\caption{function $P$}
							\label{tab:p}
						\end{table}
				\end{enumerate}
				\begin{table}[hbtp]
					\centering
					\begin{tabular}{ccccccccccccccccc}
						\multirow{4}{*}{$S_2$}
						 & 15 &  1 &  8 & 14 &  6 & 11 &  3 &  4 &  9 &  7 &  2 & 13 & 12 &  0 &  5 & 10\\
						 &  3 & 13 &  4 &  7 & 15 &  2 &  8 & 14 & 12 &  0 &  1 & 10 &  6 &  9 & 11 &  5\\
						 &  0 & 14 &  7 & 11 & 10 &  4 & 13 &  1 &  5 &  8 & 12 &  6 &  9 &  3 &  2 & 15\\
						 & 13 &  8 & 10 &  1 &  3 & 15 &  4 &  2 & 11 &  6 &  7 & 12 &  0 &  5 & 14 &  9\\
						\hline
						\multirow{4}{*}{$S_3$}
						 & 10 &  0 &  9 & 14 &  6 &  3 & 15 &  5 &  1 & 13 & 12 &  7 & 11 &  4 &  2 &  8\\
						 & 13 &  7 &  0 &  9 &  3 &  4 &  6 & 10 &  2 &  8 &  5 & 14 & 12 & 11 & 15 &  1\\
						 & 13 &  6 &  4 &  9 &  8 & 15 &  3 &  0 & 11 &  1 &  2 & 12 &  5 & 10 & 14 &  7\\
						 &  1 & 10 & 13 &  0 &  6 &  9 &  8 &  7 &  4 & 15 & 14 &  3 & 11 &  5 &  2 & 12\\
						\hline
						\multirow{4}{*}{$S_4$}
						 & 7 &  13 & 14 &  3 &  0 &  6 &  9 & 10 &  1 &  2 &  8 &  5 & 11 & 12 &  4 & 15\\
						 & 13 &  8 & 11 &  5 &  6 & 15 &  0 &  3 &  4 &  7 &  2 & 12 &  1 & 10 & 14 &  9\\
						 & 10 &  6 &  9 &  0 & 12 & 11 &  7 & 13 & 15 &  1 &  3 & 14 &  5 &  2 &  8 &  4\\
						 &  3 & 15 &  0 &  6 & 10 &  1 & 13 &  8 &  9 &  4 &  5 & 11 & 12 &  7 &  2 & 14\\
						\hline
						\multirow{4}{*}{$S_5$}
						 &  2 & 12 &  4 &  1 &  7 & 10 & 11 &  6 &  8 &  5 &  3 & 15 & 13 &  0 & 14 &  9\\
						 & 14 & 11 &  2 & 12 &  4 &  7 & 13 &  1 &  5 &  0 & 15 & 10 &  3 &  9 &  8 &  6\\
						 &  4 &  2 &  1 & 11 & 10 & 13 &  7 &  8 & 15 &  9 & 12 &  5 &  6 &  3 &  0 & 14\\
						 & 11 &  8 & 12 &  7 &  1 & 14 &  2 & 13 &  6 & 15 &  0 &  9 & 10 &  4 &  5 &  3\\
						\hline
						\multirow{4}{*}{$S_6$}
						 & 12 &  1 & 10 & 15 &  9 &  2 &  6 &  8 &  0 & 13 &  3 &  4 & 14 &  7 &  5 & 11\\
						 & 10 & 15 &  4 &  2 &  7 & 12 &  9 &  5 &  6 &  1 & 13 & 14 &  0 & 11 &  3 &  8\\
						 &  9 & 14 & 15 &  5 &  2 &  8 & 12 &  3 &  7 &  0 &  4 & 10 &  1 & 13 & 11 &  6\\
						 &  4 &  3 &  2 & 12 &  9 &  5 & 15 & 10 & 11 & 14 &  1 &  7 &  6 &  0 &  8 & 13\\
						\hline
						\multirow{4}{*}{$S_7$}
						 &  4 & 11 &  2 & 14 & 15 &  0 &  8 & 13 &  3 & 12 &  9 &  7 &  5 & 10 &  6 &  1\\
						 & 13 &  0 & 11 &  7 &  4 &  9 &  1 & 10 & 14 &  3 &  5 & 12 &  2 & 15 &  8 &  6\\
						 &  1 &  4 & 11 & 13 & 12 &  3 &  7 & 14 & 10 & 15 &  6 &  8 &  0 &  5 &  9 &  2\\
						 &  6 & 11 & 13 &  8 &  1 &  4 & 10 &  7 &  9 &  5 &  0 & 15 & 14 &  2 &  3 & 12\\
						\hline
						\multirow{4}{*}{$S_8$}
						 & 13 &  2 &  8 &  4 &  6 & 15 & 11 &  1 & 10 &  9 &  3 & 14 &  5 &  0 & 12 &  7\\
						 &  1 & 15 & 13 &  8 & 10 &  3 &  7 &  4 & 12 &  5 &  6 & 11 &  0 & 14 &  9 &  2\\
						 &  7 & 11 &  4 &  1 &  9 & 12 & 14 &  2 &  0 &  6 & 10 & 13 & 15 &  3 &  5 &  8\\
						 &  2 &  1 & 14 &  7 &  4 & 10 &  8 & 13 & 15 & 12 &  9 &  0 &  3 &  5 &  6 & 11\\
					\end{tabular}
					\caption{S-box $S_2 \sim S_8$}
					\label{tab:s}
				\end{table}
			\item
				\textbf{Key Schedule (KS)}

				KS gets the 64-bit key as input and outputs 16 48-bit sub keys.
				\begin{enumerate}
					\item
						Transform bits of the key using Permuted Choice 1 (PC-1). Apply Table \ref{tab:pc-1c} to get 28-bit $C_0$, and Table \ref{tab:pc-1d} to get 28-bit $D_0$.
						\begin{table}[hbtp]
							\centering
							\begin{tabular}{ccccccc}
								57 & 49 & 41 & 33 & 25 & 17 &  9\\
								 1 & 58 & 50 & 42 & 34 & 26 & 18\\
								10 &  2 & 59 & 51 & 43 & 35 & 27\\
								19 & 11 &  3 & 60 & 52 & 44 & 36\\
							\end{tabular}
							\caption{Permuted Choice 1 C}
							\label{tab:pc-1c}
						\end{table}
						\begin{table}[hbtp]
							\centering
							\begin{tabular}{ccccccc}
								63 & 55 & 47 & 39 & 31 & 23 & 15\\
								 7 & 62 & 54 & 46 & 38 & 30 & 22\\
								14 &  6 & 61 & 53 & 45 & 37 & 29\\
								21 & 13 &  5 & 28 & 20 & 12 &  4\\
							\end{tabular}
							\caption{Permuted Choice 1 D}
							\label{tab:pc-1d}
						\end{table}
					\item
						\label{step:shift}
						Rotate Shift $C_0$ and $D_0$ to the left to yield $C_1$ and $C_2$. The shift bit should be either 1 or 2, which is specified in Table \ref{tab:shift}.
						\begin{table}[hbtp]
							\centering
							\begin{tabular}{ccccccccccccccccc}
								Shift Times & 1 & 2 & 3 & 4 & 5 & 6 & 7 & 8 & 9 & 10 & 11 & 12 & 13 & 14 & 15 & 16\\
								Shift bits  & 1 & 1 & 2 & 2 & 2 & 2 & 2 & 2 & 1 &  2 &  2 &  2 &  2 &  2 &  2 &  1\\
							\end{tabular}
							\caption{Rotate Shift}
							\label{tab:shift}
						\end{table}
					\item
						\label{step:pc-2}
						Apply Permuted Choice 2 (PC-2) on $C_1$ and $D_1$. PC-2 transforms \{$C_1$, $D_1$\} (56-bit) to a 48-bit sub key $K_1$. The transform table is shown in Table \ref{tab:pc-2}.
						\begin{table}[hbtp]
							\centering
							\begin{tabular}{cccccc}
								14 & 17 & 11 & 24 &  1 &  5\\
								 3 & 28 & 15 &  6 & 21 & 10\\
								23 & 19 & 12 &  4 & 26 &  8\\
								16 &  7 & 27 & 20 & 13 &  2\\
								41 & 52 & 31 & 37 & 47 & 55\\
								30 & 40 & 51 & 45 & 33 & 48\\
								44 & 49 & 39 & 56 & 34 & 53\\
								46 & 42 & 50 & 36 & 29 & 32\\
							\end{tabular}
							\caption{Permuted Choice 2}
							\label{tab:pc-2}
						\end{table}
					\item Repeat Step \ref{step:shift} to Step \ref{step:pc-2} for 16 times, yielding sub keys $K_1 \sim K_{16}$.
				\end{enumerate}
		\end{itemize}
	\item
		\begin{itemize}
			\item Linear cryptanalysis tries to discover affine relations between cipher text and plain text to detect key bits.
			\item Differential cryptanalysis studies how different plain text or cipher text affects output result so as to figure out some critical
		\end{itemize}
	\item
		Double DES uses two instance of DES with two different keys, and triple DES use three instance with three different keys. Double DES is vulnerable since its security level is not $2^{112}$ with 112-bit key, but instead $2^{56}$ if meet-in-the-middle attack is used, which is very easy to break with modern technology. Triple DES has the security level of $2^{112}$, which can satisfy modern security criterion.
	\item
		Passwords stored with \texttt{passwd} on Unix systems are encrypted with DES by default, which is not safe since DES is vulnerable and easy to break today. Better solutions would be to use hash functions such as SHA256, because there is no need to decrypt a password but only a verification is required.
\end{enumerate}

\section*{Ex.4 - Programming}

See in folder \textbf{ex4}.

\end{document}
